\documentclass[12pt,a4]{article}\usepackage[]{graphicx}\usepackage[]{xcolor}
% maxwidth is the original width if it is less than linewidth
% otherwise use linewidth (to make sure the graphics do not exceed the margin)
\makeatletter
\def\maxwidth{ %
  \ifdim\Gin@nat@width>\linewidth
    \linewidth
  \else
    \Gin@nat@width
  \fi
}
\makeatother

\definecolor{fgcolor}{rgb}{0.345, 0.345, 0.345}
\newcommand{\hlnum}[1]{\textcolor[rgb]{0.686,0.059,0.569}{#1}}%
\newcommand{\hlstr}[1]{\textcolor[rgb]{0.192,0.494,0.8}{#1}}%
\newcommand{\hlcom}[1]{\textcolor[rgb]{0.678,0.584,0.686}{\textit{#1}}}%
\newcommand{\hlopt}[1]{\textcolor[rgb]{0,0,0}{#1}}%
\newcommand{\hlstd}[1]{\textcolor[rgb]{0.345,0.345,0.345}{#1}}%
\newcommand{\hlkwa}[1]{\textcolor[rgb]{0.161,0.373,0.58}{\textbf{#1}}}%
\newcommand{\hlkwb}[1]{\textcolor[rgb]{0.69,0.353,0.396}{#1}}%
\newcommand{\hlkwc}[1]{\textcolor[rgb]{0.333,0.667,0.333}{#1}}%
\newcommand{\hlkwd}[1]{\textcolor[rgb]{0.737,0.353,0.396}{\textbf{#1}}}%
\let\hlipl\hlkwb

\usepackage{framed}
\makeatletter
\newenvironment{kframe}{%
 \def\at@end@of@kframe{}%
 \ifinner\ifhmode%
  \def\at@end@of@kframe{\end{minipage}}%
  \begin{minipage}{\columnwidth}%
 \fi\fi%
 \def\FrameCommand##1{\hskip\@totalleftmargin \hskip-\fboxsep
 \colorbox{shadecolor}{##1}\hskip-\fboxsep
     % There is no \\@totalrightmargin, so:
     \hskip-\linewidth \hskip-\@totalleftmargin \hskip\columnwidth}%
 \MakeFramed {\advance\hsize-\width
   \@totalleftmargin\z@ \linewidth\hsize
   \@setminipage}}%
 {\par\unskip\endMakeFramed%
 \at@end@of@kframe}
\makeatother

\definecolor{shadecolor}{rgb}{.97, .97, .97}
\definecolor{messagecolor}{rgb}{0, 0, 0}
\definecolor{warningcolor}{rgb}{1, 0, 1}
\definecolor{errorcolor}{rgb}{1, 0, 0}
\newenvironment{knitrout}{}{} % an empty environment to be redefined in TeX

\usepackage{alltt}

% ---- Metadata ---- %

\title{Honesty by Convenience: Corruption Tolerance in Ecuador}
\author{Daniel Sánchez}
\date{June 2022}

% ---- Load Packages ---- %

% Math

\usepackage{savesym} % Need to "save" the command that is already defined \varTheta

\usepackage{amsmath}
  \savesymbol{varTheta} 

% Fonts

% To set the TNR font for both text and equations:

\usepackage{mathspec}
  \setallmainfonts(Digits,Greek,Latin){Times New Roman}
\restoresymbol{MTP}{varTheta}

% Formatting

\usepackage{setspace}
  \doublespacing

\usepackage[margin = 1in]{geometry}

\usepackage{lscape}

% Setting the size of the section titles

\usepackage{titlesec}

\titleformat*{\section}{\normalsize\bfseries}

% Citation & Bibliographies

\usepackage[backend = biber, style = apa, citestyle = apa]{biblatex}
  \addbibresource{refs.bib}
  
% For tables:

 % For the modelsummary tables:
\usepackage{siunitx}
\usepackage{booktabs} 
  \newcolumntype{d}{S[input-symbols = ()]}
\usepackage{multirow}
\usepackage[flushleft]{threeparttable}

% For figure and table captions

\usepackage{caption}
  \captionsetup{labelfont = bf} % All in bold  
  
% Other packages

\usepackage{csquotes} % For quotation marks

\usepackage{epigraph} % For epigraph
  \setlength\epigraphwidth{9cm}
  \setlength\epigraphrule{1pt}

\usepackage{float} % For the H float option- only used in emergencies (lol)

\usepackage{textcomp} % For the registered trademark symbol.

% Always load these packages at the end of the preamble:

\usepackage{hyperref}

% ---- R Stuff to be used in the whole document ----

% Here I will execute or source R code through chunks that I need to use throughout the whole document.

% General settings



% Load the data by sourcing the data manipulation script. Note that survey design objects are indeed created in this script.
% We use the time argument in the chunks to reread or rerun the chunk in case external files are updated and chunks need to be rerun and re-cached.


% Perform all survey-robust tabulations by sourcing the R Script. 
% These are used on the text later.


% Run the first models


\IfFileExists{upquote.sty}{\usepackage{upquote}}{}
\begin{document}

% ---- Sections ---- %

% Abstract Child Document


% Abstract .Rnw File


\begin{center}
\textbf{
Honesty by Convenience: Corruption Tolerance in Ecuador\\
Daniel Sánchez}
\end{center}

\textit{
Attitudes towards corruption may be a strong determinant of its incidence. Using survey
data from the AmericasBarometer (AB), binary-outcome empirical models are estimated to discover
the key determinants of an increase in corruption tolerance in Ecuador between 2014 and
2016. It is found that two key variables may have influenced this increase. People who approved
of the President’s job performance initially justified bribes less, but by 2016 supporters justified corruption more. Also, those who identified closer to the political right justified corruption more in 2016. The jump is explained through these variables as the percentage of people who approved of the President decreased and the percentage of people identifying with the right increased. It is also found that the people who were either employed or outside the labor force justified bribes more in 2016 when compared to those who were unemployed.}

% Introduction Child Document


% Introduction .Rnw File


\section{Introduction}
\enquote{Even if you are from [my political party], I will fulfill my duties. If you steal, steal well!  Justify well! But do not let your affairs be seen, comrades\footnote{Translated from Cerda, 2021 in \cite[para. 2]{PlanV.2021}}}. Uttered publicly by Rosa Cerda, Ecuadorian congresswoman for the Napo province \parencite{Castro.2021}, these comments met widespread criticism around the country, although the remarks were initially met by cheers from the audience she addressed. However, Cerda's declarations did not transcend an eight day suspension \parencite{Ordonez.2021} and the whole event was soon forgotten by most citizens. 

This episode is only one of many corruption-related scandals that have happened in Ecuador, a middle-income country in South America. The country has seen increased COVID-19 vaccine inequality \parencite{Taj.2021}, weakened public health services \parencite{Celi.2020}, policymakers charging fees for political positions \parencite{Espinosa.2021}, lost Social Security funds \parencite{Pesantes.9152020}, a convicted former president as well as two vice-presidents impeached and removed on charges of corruption \parencite{Cabrera.2020}, among others. However, it is almost as if these no longer cause outrage: at most, they cause a sigh of disappointment or social media outrage which dwindles shortly after.

This apparent ambivalence has seen Ecuador place well above the corruption median in the world according to both Transparency International's and the World Bank's corruption indexes. About 90\% of voting-age Ecuadorians believe that at least half of politicians are corrupt and more than a quarter of them admit having been involved with bribes in 2019, according to the AmericasBarometer (AB) survey, by the Latin American Public Opinion Project (LAPOP). However, a mere 8.08\% consider that corruption is the most serious problem faced by the country and in fact 25.38\% of Ecuadorians believe that paying a bribe is justified. Further, tolerance to corruption has risen 11.79 percentage points from 2014 to 2019. Furthermore, Figure \ref{fig:ctolmap} shows that Ecuador is also one of the countries with the highest corruption tolerance in the region.

% Corruption Tolerance Choropleth Map based on 2019 values

% Sorry, this map was done with the paid AmericasBarometer databases, so I cannot actually put my source code and have it executed by KNITR. USFQ students are free to access another version of the document, they must only email me. Third parties must wait until I do it with the free databases. Check out hbc-v2 for some more info.

\begin{figure}[htbp]
  \label{fig:ctolmap}
      \caption{Corruption Tolerance (\%) Choropleth Map in 2019}
    \begin{center}
    \includegraphics{images/ctol_map.pdf}
    \end{center}

A choropleth map showing corruption tolerance percentages across Latin America in 2019, where Ecuador places third in the most corruption tolerant countries. Darker areas imply higher percentages of corruption tolerance. Figure prepared by the author with data from the \textregistered AmericasBarometer 2018/19. 
\end{figure}

This paper aims to investigate the determinants of the largest corruption tolerance increase in Ecuador, seen from 2014 to 2016, as it can be seen in Figure \ref{fig:ctoly}. This period coincided with two key events in the country. First, the popularity of the governing regime sharply dropped for the first time in a decade \parencite{Quillupangui.2016}. Second, the country faced an economic recession \parencite{Weisbrot.2017}. The present paper will seek to relate  This is done by estimating a binary-outcome model through logistic regression, which relates the probability of tolerating corruption to several individual-level public opinion and economic indicators using the data from the AB. It is determined that changes in presidential job approval as well as in political wing preferences during the 2014 and 2016 period could have influenced the corruption tolerance increase. It is also found that those not unemployed justified corruption more in 2016 relative to those who were unemployed.

% Corruption Tolerance Time Series Graph

\begin{figure}[htbp]
\label{fig:ctoly}
\caption{Percent of Ecuadorians who justify corruption, by year}
\begin{knitrout}
\definecolor{shadecolor}{rgb}{0.969, 0.969, 0.969}\color{fgcolor}

{\centering \includegraphics[width=\maxwidth]{figure/ctol_graph-1} 

}


\end{knitrout}


The evolution of corruption tolerance for Ecuador. The largest increase is seen from 2014 to 2016. Error bars show 95\% confidence intervals, considering survey design effects. Figure prepared by the author, with the open-access AB data.
\end{figure}

Changes to the attitudes toward corruption can be important for studying corruption incidence. A higher degree of corruption tolerance will eventually lead to larger corruption environments \parencite{Campbell.2014}. Learning what drives corruption tolerance can then foster better policymaking and citizen attitudes which steers individuals away from becoming involved in corruption.

The rest of the paper proceeds as follows. The following section gives an institutional and historical background of the paper's setting, Ecuador. Section 3 reviews the relevant literature. Section 4 develops the empirical methodology. Section 5 reviews the results from the empirical estimation. Section 6 discusses these results, and Section 7 concludes. 

% Institutional Background Child Document


% Context .Rnw File

\section{Institutional and Historical Background}
\label{sec:background}

Ecuador is a small middle income country located in upper South America next to Colombia and Peru. Its GDP for 2022 is projected to be 115.47 billion US dollars, with an expected growth rate of 2.68\% \footnote{Data is from the IMF's World Economic Outlook dataset for October 2022.}

\textcite{Adoum.2000} describes how Ecuadorian citizens tend to surrender to an inefficient political system and to dishonesty: they recognize the system as corrupt but still do not fully condemn it. From this, Adoum suggests that the feeling of impotence within a corrupt system makes average citizens feel that the law is illegitimate and thus break it whenever it suits them without any kind of remorse. This is an interesting application of \textcite{Ashforth.2003} who suggest that dishonest behavior is rationalized by rejecting the legitimacy of authorities or view corruption as revenge for unfair acts. Adoum also confirms the idea that citizens who engage in dishonest behavior do not perceive themselves as corrupt by pointing to how detached the average Ecuadorian feels from the political process. Likely, this makes it easier to engage in \enquote{petty} dishonest acts.

\textcite{Hurtado.2007} claims that the Ecuadorian society has been historically prone to dishonest behaviors as a result of low economic development, feudalism in early Ecuadorian settlements, the effect of Spanish hierarchical culture, racism, extreme catholicism, inmoderate collectivism, among others. Hurtado holds that for a long time the Ecuadorian society has functioned with unfair and dishonest social and economic mechanisms, like political clientelism and nepotism for job hirings, disrespect to property rights, non-compliance with social and legal contracts, etc. It is suggested that these practices have hindered social and economic mobility, especially for historically marginalized ethnic groups. This makes dishonest behaviors even more widespread: a self-fulfilling prophecy of dishonest and pervasive behavior. 

\textcite{Loaiza.2019} provides a recent account of some of the mentioned author's claims. \textcite{Loaiza.2019} cites the Latinobarómetro survey, which finds that 44\% of Ecuadorians \enquote{are willing to accept crimes against the public administration - in other words, corruption- in exchange for basic services, public buildings or roads} (Loaiza, 2019, para. 5). This estimate places Ecuador as the sixth most tolerant country to corruption in Latin America. The survey confirms the findings of the AB, as it concurs in how Ecuadorians do not consider corruption as the most important problem and how it is perceived to be very widespread. The author also suggests a self-interest theory to justifying corruption, as about 11\% believe that it is better to be an accomplice of corruption than to denounce it. 

As pointed out previously, the corruption tolerance increase happened at the same time as other key events. First, AB indicators denote a political crisis, as support for President Rafael Correa's regime took a sharp hit. Second, a recession hit Ecuador due to a commodity price collapse, an earthquake and other circumstances. Below, Figure \ref{fig:ecua_pol} shows several public opinion indicators and Figure \ref{fig:ecua_ec} displays economic conditions, both observed and perceived from 2014 to 2019. 
% Create the data to be used for the political opinion variables:

% Now do the graph
\begin{figure}[htbp]
\centering
\fbox{
\begin{minipage}{\textwidth}
\caption{Ecuadorian public opinion indicators, 2004-2019}
\label{fig:ecua_pol}
\begin{knitrout}
\definecolor{shadecolor}{rgb}{0.969, 0.969, 0.969}\color{fgcolor}

{\centering \includegraphics[width=\maxwidth]{figure/political_graph-1} 

}


\end{knitrout}
\textbf{Note:} The graph shows time series for political public opinion questions asked in the AB. Percentages are estimated as explained in \hyperref[app:first]{Appendix A} and error bars show 95\% confidence intervals considering design effects. Figure prepared by the author. \end{minipage}
}
\end{figure}

The AB data shows that indeed the President reached an all-time high popularity in 2014 and then a severe drop in 2016. This is seen through the percent of people who approve the President's job performance and the percent who report confidence in him. Another notable change in the political landscape is the way that voting-age population politically identified. There was a strong increase of the people who identified as the \enquote{right}, while those who identified with the \enquote{left} did not see significant changes.

Regarding the economic recession, \textcite{Orozco.2015} holds that although the commodity price collapse in 2008 was greater, there was little reduction in economic activity as the country had greater possibilities of international financing and savings left over from past oil funds, which were used to keep government expenditure high. In 2016, as savings eroded and government debt had grown bigger, the economy stagnated significantly for the first time in the Correa administration. Combined with the lack of competitiveness in exports due to US dollar appreciations and the poor public finance administration \parencite{Hurtado.2018}, the country fell into a deep economic recession. While the official GDP figures may show only a small reduction in GDP growth, \textcite{Hurtado.2018} holds that these figures are overestimated.

% Here I create my graph, but I need to load some libraries first and create the data needed for my graphs.

% Now I do the graph:
\begin{figure}[htbp]
\begin{center}
\fbox{
\begin{minipage}{\textwidth}
\caption{Ecuadorian economic conditions 2004-2019}
\label{fig:ecua_ec}
\begin{knitrout}
\definecolor{shadecolor}{rgb}{0.969, 0.969, 0.969}\color{fgcolor}

{\centering \includegraphics[width=\maxwidth]{figure/econ_graph-1} 

}


\end{knitrout}
\textbf{Note:} Time series line graphs showing key economic indicators for the country between 2004 and 2019. Real GDP growth and unemployment rates extracted from the World Bank's World Development Indicators. WTI oil barrel prices extracted from FRED. The rest are estimates computed with the open-access AB databases, which include 95\% confidence intervals adjusted for design effects. See \hyperref[app:first]{Appendix A} for details on calculations. Figure prepared by the author. 
\end{minipage}}
\end{center}
\end{figure}

Figure \ref{fig:ecua_pol} shows several indicators of public opinion in the country. The AB data shows that indeed the President reached an all-time high popularity in 2014 and then a severe drop in 2016. This is seen through the percent of people who approve the President's job performance and the percent who report confidence in him. Another notable change in the political landscape of this period is the way that voting-age population identified politically. There was a notable increase of the people who identified as the \enquote{right} of the political wings, while those who identified with the \enquote{left} did not see significant changes.

% Literature Review Child Document


% Literature Review Rnw File


\section{Literature Review}

The literature on corruption mostly focuses on corruption incidence and how it may determine other economic and social outcomes. Lesser attention has been given to the corruption tolerance phenomenon, yet a key finding of this reduced subset of the literature is that the more tolerance and exposure to corrupt acts, the more likely it is that they will spread across individuals. \textcite{Ariely.2019} discuss experimental findings which show that individuals that pay a bribe or are requested to pay one are more likely to behave dishonestly in subsequent ethical dilemmas. Further experimental evidence from \textcite{Gino.2009} also shows that subjects with more exposure to dishonest behaviors are more likely to engage in it. An empirical study of corrupt organizations by \textcite{Campbell.2014} shows that corrupt acts create an organizational culture which fosters the incidence of corruption among its members. The corrupt culture may change the behavior of otherwise honest individuals through social pressure, notably when philosophies which hold that \enquote{the ends justify the means}. These findings suggest that social norms and outer circumstances shape the way in which corruption is interpreted by the members of organizations, which in turn influence how corruption might appear. It becomes key to understand the determinants of corruption tolerance.

It is adequate to place corruption in a basic framework which will inform the way that people behave around it. I build a framework based on the implications of the microeconomic modelling of corruption and on the organizational behavior of corrupt acts proposed by \textcite{Ashforth.2003}. The framework allows me to determine that corrupt acts can be shaped by economic and social payoffs, where three key mechanisms influence social payoffs, namely, behaviors of rationalization, institutionalization and socialization of corruption. Through this famework, it becomes apparent that any kind of study of corruption tolerance determinants should take into account indicators that can account for both economic and social determinants. 

\textcite{Shleifer.1993} model bribes in a way where a public official trades public goods in exchange for bribes. Private agents then pay them to receive the good and the consumer surplus that any transaction brings. This might be understood as an individual economic incentive to engage in corrupt acts: paying the bribe allows the use of a desirable public good, or allows for quicker access to it. Thus, economic convenience could be an important determinant of how people behave around corruption: people may tolerate dishonesty if it means a positive economic payoff. 

On the other hand, there might be also moral considerations to the decision of tolerating or engaging in corruption. While the economic payoff of paying or receiving a bribe may be positive, the moral connotation of the act may bring shame or rejection from society. Avoiding a bad image may very well become an important determinant of the decision of engaging in a corrupt act. Nevertheless, in environments where corruption is tolerated the negative social payoff of bribing might be smaller, which increases incentives for being corrupt. The importance of social payoffs for economic transactions cannot be neglected. \textcite{Akerlof.1980} holds that these might change economic outcomes in a significant way, deviating from the equilibria derived from the assumptions of rational self-interested behavior. It then becomes key to understanding how the social payoffs of corrupt acts are determined, as it could be assumed that most of the time the economic payoff of bribes is positive for the corrupt individual. 

\textcite{Ashforth.2003} develop a model to explain how corruption is normalized or tolerated in an organization, which helps to understand how these social payoffs are determined. They argue that after an initial exposure to corruption brought by several environmental factors, the corrupt decision starts being used in the future by organization members. Corrupt behavior then becomes part of the organizational culture or becomes \textit{institutionalized}, as these acts start to be considered as routine for the organization. In the present context, the organizational culture may enclose the complete political apparatus of a country but also the more diffuse organizations that political affiliations represent. 

Leadership in the organization is crucial for institutionalization behaviors that determine the social payoffs of corruption. Leaders need not engage in corrupt acts themselves to foster their normalization, they can simply facilitate or ignore the initial corrupt acts to have subordinates start normalizing corruption. Moreover, strong rewards or punishments for engaging or not engaging in corrupt acts, as well as a strong emphasis on results also may lead to the institutionalization of corruption. Subordinates do not second-guess their superiors' decisions as a result of the habit of obedience, which is more prevalent in highly hierarchical organizations. The authors also note that the psychological process of obedience also comes with a sense of helplessness and resignation, where the subordinate becomes detached from the moral dilemma by thinking that they only follow orders. 

Along with the institutionalization of corrupt acts, two other mechanisms are involved in the normalization of corruption. Together, these mechanisms reinforce each other so that individuals in corrupt organizations do not believe they are really corrupt when engaging in dishonest acts, which in turn fosters corruption further. The \textit{rationalization} mechanism of corruption is especially important, as it can be easily modelled at the individual level. The authors argue that corrupt individuals rationalize corruption in a way that they \enquote{avoid the adverse effects of an undesirable social identity} \parencite[p.13]{Ashforth.2003}. Rationalization is based on the behavioral premise that the members of an organization may try to resolve the ambiguity that surrounds action in a way that it serves their own interests. 

There are several ways through which the mechanism of rationalization appears. One of them is the \textit{denial of responsibility}, in which corrupt individuals convince themselves that they have no other choice than to engage in corrupt acts due to external circumstances. The authors also consider the case when individuals see their own corruption as a form of revenge against unfair or corrupt acts done to them. A related type of rationalization is when corrupt acts are justified because the actors perceive those that denounce corruption as illegitimate or hypocritical authorities, charged with motives other than the well-being of the organization.

The socialization mechanism is concerned with the peer effects of corruption, where dishonest practices are \enquote{taught} to organization newcomers. Newcomers will be initially induced to change their attitudes towards corrupt beliefs, then being peer-pressured to escalate these practices. Since newcomers strive to be accepted, they end up adopting these dishonest behaviors as their own, while they also rationalize it to avoid the social costs of being dishonest. Later the newcomers become the ones that exert peer pressure on future members.

Having established a framework which will allow for better modelling of corruption tolerance, it is useful to look at what the literature has found with the variable. Corruption tolerance in the AB has been studied for several countries in the Latin American region. \textcite{Singer.2016} find that for every country in Latin America in 2014, at least 60\% of the respondents perceive their governments to be corrupt but a much smaller proportion considers corruption to be the most important problem in their countries. It is found that those who justify corruption are those who have been exposed to some kind of bribe in the past\footnote{The original wording by the authors in the AB reports is \textit{corruption victimization}. Here, this variable is referred to as \textit{corruption exposure}, to account for the possibility that the respondent can be either a victim of corruption by being forced to pay a bribe or the initial corrupt agent who offers to pay one.}. Other significant determinants of corruption tolerance in 2014 were age and the urban-rural dichotomy. Younger participants tend to justify corruption to a higher degree, a finding robust through time and across countries of the region. Those living in rural settings also tend to justify corruption more.

\textcite{Lupu.2017} shows that corruption tolerance has been growing consistently in the region and that the average Latin American country has about a fifth of its population believing that corruption is justified. Between 2014 and 2016, corruption tolerance grew from 17.4\% to 20.5\% in the region. It is found that older citizens as well as those exposed to corruption before are more prone to justify it. The level of perceived corruption also apperas as a significant determinant. \textcite{Lupu.2017} also arrive to the worrying conclusion that corruption may have become a \enquote{a self-fulfilling prophecy: as more and more citizens perceive that corruption is more widespread, they also become more likely to condone it}(p. 67). 

Finally, regarding Ecuadorians' corruption tolerance behaviors, \textcite{Moscoso.2018} find that corruption is perceived to be very widespread in the country yet it is not regarded as an important. It is also noted that for 2016, Ecuador became one of the countries which was the most tolerant of corruption in the region. \textcite{Montalvo.2019} finds that the general trend for younger people to justify corruption more is also found in Ecuador. For the same round, \textcite{Moscoso.2020} find that besides age, interest in politics is a statistically significant predictor as well as exposure to corruption, as found by \textcite{Lupu.2017}. According to these authors, the empirical evidence may very well support the fact that corruption has become a known inconvenience for daily Ecuadorian life rather than an unacceptable threat to the system, and that it is endemic to the political and social environments. 

% Methodology Child Document


% Methodology .Rnw File


\section{Methodology}
\label{sec:methodology} % Label the section to cross-reference later.

The AmericasBarometer (AB) survey from the Latin American Public Opinion project is used in this paper to investigate the corruption tolerance increase in Ecuador. This survey was administered in Ecuador and several other Latin American countries from 2004 to 2019, at mostly two-year intervals. It asks about public opinion matters, including democracy, corruption, among others. The open-access AB databases available in the LAPOP \href{https://www.vanderbilt.edu/lapop/data-access.php}{website} are used for the empirical models. Table \ref{tab:descrip} presents descriptive statistics for all variables used.

% Here I will create a descriptive table including averages (and proportions) by year.
% This table is a bit difficult to construct because of its survey-weighted statistics and the rather complex structure, which is why I created it using Excel, exporting the results from calculations done in R.
\begin{table}[htbp!]
\onehalfspacing
\begin{center}
\caption{Descriptive statistics for all variables}
\label{tab:descrip}
\begin{tabular}{llcccc}
\toprule
\multicolumn{1}{c}{\multirow{2}{*}{Variable}} & \multirow{2}{*}{\begin{tabular}[c]{@{}l@{}} Question code \end{tabular}} & \multicolumn{2}{c}{2014}  & \multicolumn{2}{c}{2016}  \\ 
\cmidrule(l{3pt}r{3pt}){3-4} \cmidrule(l{3pt}r{3pt}){5-6}
\multicolumn{1}{c}{}                          &                                                                                         & Est. & SE & Est. & SE \\ \midrule
Corruption tolerance                          & EXC18                                                                                   & 13.59    & 1.39           & 27.18    & 1.21           \\
Unemployment                                  & OCUP4A                                                                                  & 10.06    & 1.04           & 22.89    & 1.2            \\
Confidence in the President                   & B21A                                                                                    & 69.01    & 1.77           & 49.64    & 1.49           \\
Approval of the President                     & M1                                                                                      & 70.26    & 1.57           & 55.41    & 1.43           \\
Economic situation (Worse)                            & IDIO2                                                                                   & 22.93    & 1.26           & 51.76    & 1.45           \\
No political wing                             & L1                                                                                      & 21.49    & 2.11           & 8.67     & 0.74           \\
Center                                        & L1                                                                                      & 42.58    & 1.92           & 45.7     & 1.49           \\
Left                                          & L1                                                                                      & 22.23    & 1.25           & 22.46    & 1.24           \\
Right                                         & L1                                                                                      & 13.7     & 1.16           & 23.17    & 1.15           \\
Women                                         & Q1                                                                                      & 50.37    & 0.34           & 50.29    & 0.3            \\
Age                                           & Q2                                                                                      & 39.41    & 0.17           & 38.64    & 0.22           \\
Years of education                            & ED                                                                                      & 10.67    & 0.15           & 11.43    & 0.14           \\
Urban                                         & UR                                                                                      & 65.21    & 4.11           & 66.41    & 4.07           \\
External political efficacy                   & EFF1                                                                                    & 35.31    & 1.69           & 41.93    & 1.33           \\
Internal political efficacy                   & EFF2                                                                                    & 38.55    & 1.58           & 41.49    & 1.34           \\
Participated in a protest                     & PROT3                                                                                   & 6.82     & 0.89           & 4.67     & 0.55           \\
Interest in politics & POL1 & 33.45 & 1.63 & 32.29 & 1.35 \\
Perceives corruption                          & EXC7, EXC7NEW                                                                           & 70.29    & 1.74           & 83.49    & 0.97           \\
Exposed to corruption                         & EXC 2,6,11,13,14,15,16                                                                  & 26.97    & 2.01           & 27.69    & 1.23           \\ 
\bottomrule
\end{tabular}
\end{center}
Descriptive statistics table with estimates (Est.) and robust standard errors (SE), where age, years of education and the external and internal political efficacies are arithmetic means. All other variables are percentages, calculated for 2014 and 2016. Standard errors are adjusted for survey-design effects. Data is from the open-access AmericasBarometer.
\end{table}

The empirical models estimated in this study will use the 2014 and 2016 rounds of the AB in Ecuador, with $n_{2014}=1489$ and $n_{2016}= 1545$. The survey is based on a multi-stage national probability design. The survey-design adjusted errors for each of these surveys are $\pm 2.5\%$ and $\pm 1.9\%$, respectively (\cite{LAPOP.2014}; \cite{LAPOP.2017}). Both surveys are self-weighted, however, 95\% confidence intervals for the descriptive statistics which are adjusted for survey-design effects are presented when relevant. 

The empirical analysis is concerned with the \emph{EXC18} question: \enquote{Do you think given the way things are, sometimes paying a bribe is justified?} \parencite[p.96]{Moscoso.2018}, originally asked in Spanish. The question has been asked in all survey rounds in Ecuador and is the last one after a set of questions regarding corruption exposure and perception. This variable ($ctol$) is equal to 1 when the respondent answers \enquote{Yes}, 0 when the answer is \enquote{No} and dropped from the model otherwise. All models have $ctol$ as the explained variable and responses to other questions are used as regressors. 

In order to identify the changes in behavior which led to the increase, the survey rounds are pooled and the following general model is estimated: 

\begin{equation}
\label{eqn:genmod}
P(ctol = 1 | \textbf{\textit{X}} \hspace{0.04cm}) = G (\textbf{\textit{X}} \theta ) = G \left[ \beta_0 + \delta_0 y_{16} + \textbf{\textit{R}}'\beta + \delta_1 (y_{16} \cdot x^*) \right]
\end{equation}

where $\textbf{\textit{R}}$ is a vector of controls and $x^*$ is a key regressor whose change across time may have significantly influenced the rise of $ctol$ between 2014 and 2016. This key regressor is interacted with a year dummy, $y_{16}$, which equals unity for 2016 observations. The complete regressors' vector $\textbf{\textit{X}}$ includes all variables in $\textbf{\textit{R}}$, the key regressor $x^*$ and the interaction term. The parameters vector $\theta$ includes the vector $\beta$ as well as $\beta_0$, $\delta_0$ and $\delta_1$. $G$ is the link function; in this paper I follow the literature and use a logistic function as $G$. 

Consider the partial effect of the key regressor $x^*$ on $P(ctol =1| \textbf{\textit{X}})$:
\begin{equation}
\label{eqn:keype}
\dfrac{\partial P(ctol = 1 | \textbf{\textit{X}} \hspace{0.04cm})}{\partial x^*} = \dfrac{\partial G}{\partial \theta} \cdot 
\dfrac{\partial \theta}{\partial x^*} = \dfrac{\partial G}{\partial \theta} \cdot (\beta_{x^*}+ \delta_1 y_{16})
\end{equation}

The parameter $\delta_1$ would then measure the ceteris paribus effect of a change in the key regressor $x^*$ from 2014 to 2016 in $ctol$. Therefore, the coefficient of interest in this study is $\widehat{\delta}_1$. If there has been a change in 2016 in $x^*$ which significantly influences corruption tolerance, $\widehat{\delta_1}$ should be statistically significant. Further, a $\widehat{\delta}_1$ coefficient not statistically different from zero would mean that individuals with and without this key characteristic are equally likely to justify corruption across time. 

Average partial effects tables are shown for all models. I use survey-weighting to adjust for complex-survey design effects, as suggested by \textcite{Castorena.2021}. Since the sample is self-weighted, survey-weighting does not affect magnitudes, only standard errors.

% Results 1 Child Document


% Results I .Rnw File

\section{Results}
\label{sec:results}
Section \ref{sec:background} identified some key events which happenned at the same time of the corruption tolerance increase. Two economic variables observed at the individual level through the AB significantly changed during this period: the percent of people who report a worse economic situation as well as unemployment. Variables which proxy attitudes in the political landscape have also significantly changed: the percentage of people who confide in the President, the percentage who approve the President's job and also the percentage of people who identify with the political right wing. These variables where used for simple empirical models, which follow the equation below.

\begin{equation}
\label{eqn:simplemod}
P(ctol = 1 | \textbf{\textit{X}} \hspace{0.04cm}) = G \left[ \beta_0 + \delta_0 y_{16} + \beta_1 x^* + \delta_1 (y_{16} \cdot x^*)\right]
\end{equation}
where the key regressor $x^*$ can be: a dummy variable set to unity for respondents who answered that their economic situation is worse (Model 1), a dummy variable set to unity for those who report being unemployed (Model 2), a discrete variable with numbers 1-7, where higher values imply a higher degree of confidence in the President (Model 3), a discrete variable with numbers 1-5, with higher numbers implying a higher rating of the President's job performance (Model 4) or a discrete variable with numbers from 1-10 where 1 is the extreme left and 10 is the extreme right (Model 5). 

Table \ref{tab:simplemodel} presents coefficients of the logistic model for Equation \ref{eqn:simplemod} and Table \ref{tab:apesimp} presents their associated average partial effects. It is show that an unemployed person is 5.9\% more likely to justify corruption. Additionally, a respondent who answered one number higher for an increased degree of confidence in the President was 2.4\% less likely to justify corruption. Finally, a person who rated the President's job performance one unit higher was 4.4\% less likely to justify corruption. All other partial effects are not significant.

% Now, I'll make the table with modelsummary from the sourced stuff. 
\begin{table}[htbp!]
\caption{Logit coefficients for baseline models}
\label{tab:simplemodel}

\begin{tabular}[t]{lccccc}
\toprule
  & Model 1 & Model 2 & Model 3 & Model 4 & Model 5\\
\midrule
Constant & \num{-1.894}*** & \num{-1.989}*** & \num{-0.455}** & \num{0.553} & \num{-1.527}***\\
 & (\num{0.127}) & (\num{0.110}) & (\num{0.208}) & (\num{0.362}) & (\num{0.196})\\
2016 Dummy & \num{0.848}*** & \num{1.001}*** & \num{-0.188} & \num{-1.251}*** & \num{0.278}\\
 & (\num{0.158}) & (\num{0.132}) & (\num{0.238}) & (\num{0.415}) & (\num{0.234})\\
Worse Economic Situation & \num{0.131} &  &  &  & \\
 & (\num{0.169}) &  &  &  & \\
Unemployment &  & \num{1.015}*** &  &  & \\
 &  & (\num{0.205}) &  &  & \\
Confidence in President &  &  & \num{-0.288}*** &  & \\
 &  &  & (\num{0.037}) &  & \\
Approval of Pres. Performance &  &  &  & \num{-0.648}*** & \\
 &  &  &  & (\num{0.096}) & \\
Political Wing &  &  &  &  & \num{-0.047}\\
 &  &  &  &  & (\num{0.038})\\
Econ. Situation Interaction & \num{-0.025} &  &  &  & \\
 & (\num{0.197}) &  &  &  & \\
Unemployment Interaction &  & \num{-1.005}*** &  &  & \\
 &  & (\num{0.256}) &  &  & \\
Pres. Confidence Interaction &  &  & \num{0.206}*** &  & \\
 &  &  & (\num{0.044}) &  & \\
Pres. Approval Interaction &  &  &  & \num{0.568}*** & \\
 &  &  &  & (\num{0.111}) & \\
Pol. Wing Interaction &  &  &  &  & \num{0.095}**\\
 &  &  &  &  & (\num{0.043})\\
\midrule
$N$ & \num{2948} & \num{2950} & \num{2944} & \num{2941} & \num{2535}\\
AIC & \num{2893.64} & \num{2889.04} & \num{2848.57} & \num{2844.82} & \num{2574.81}\\
BIC & \num{2926.37} & \num{2920.98} & \num{2881.80} & \num{2876.65} & \num{2606.10}\\
\bottomrule
\end{tabular}


\vspace{0.15cm}
Logit coefficients of baseline models (Equation \ref{eqn:simplemod}) with design-adjusted std. errors. *$p$ < 0.1, **$p$< 0.05, ***$p$ < 0.01.
\end{table}

Consider the logit coefficients in Table \ref{tab:simplemodel}. The coefficient for the year dummy confirms the significance of the corruption tolerance increase in 2016, which is lost when considering interaction terms with confidence in the President, and actually has a negative sign with the other political variables. The inclusion of unemployment and economic situation do not eliminate the significance of the year dummy. Model 1 suggests that a person who reports having a worse economic situation does not tolerate corruption differently than those who report a same or equal economic situation. According to Model 2, respondents who were unemployed were more likely to justify corruption than those who were not unemployed\footnote{In this case, not being unemployed means either being employed, salary and hours worked notwithstanding, and also not being in the labor force (students, rentists, among others). Results all throughout this paper are robust to including an employment variable.} The interaction term in this model has a negative sign, which shows that the effect of unemployment in 2016 was less than the effect in 2014, meaning unemployed people actually justified corruption less after political instability set in. 

% Do the APE table
\begin{table}[htbp!]
\caption{Average partial effects for logit models in Table \ref{tab:simplemodel}}
\label{tab:apesimp}

\begin{tabular}[t]{lccccc}
\toprule
  & Model 1 & Model 2 & Model 3 & Model 4 & Model 5\\
\midrule
2016 Dummy & \num{0.131}*** & \num{0.126}*** & \num{0.109}*** & \num{0.117}*** & \num{0.124}***\\
 & (\num{0.020}) & (\num{0.019}) & (\num{0.019}) & (\num{0.019}) & (\num{0.020})\\
Worse Economic Situation & \num{0.018} &  &  &  & \\
 & (\num{0.014}) &  &  &  & \\
Unemployment &  & \num{0.058}*** &  &  & \\
 &  & (\num{0.020}) &  &  & \\
Confidence in President &  &  & \num{-0.024}*** &  & \\
 &  &  & (\num{0.003}) &  & \\
Approval of Pres. Performance &  &  &  & \num{-0.044}*** & \\
 &  &  &  & (\num{0.008}) & \\
Political Wing &  &  &  &  & \num{0.003}\\
 &  &  &  &  & (\num{0.003})\\
\midrule
$N$ & \num{2948} & \num{2950} & \num{2944} & \num{2941} & \num{2535}\\
\bottomrule
\end{tabular}


\vspace{0.15cm}
Average partial effects for models in Table \ref{tab:simplemodel}, with design-adjusted std. errors. *$p$ < 0.1, **$p$< 0.05, ***$p$ < 0.01.
\end{table}

Models 3 and 4 display the same relationship: people who either trust or approve of the President in a higher degree also tolerate corruption less. A more zealous supporter of the regime believed bribes were not justified; however, this appears to change for 2016. The interaction terms for both variables are significant and positive: in 2016 supporters started to justify corruption more. This could explain the jump in corruption tolerance as regime support eroded in 2016, which meant that the number of non-supporters was higher and these respondents justified corruption more than supporters. Also, the supporters that remained started to justify bribes to a higher degree. In Model 3, the significance of the year dummy is lost, while in Model 4 its sign reversed.

The coefficients in Model 5 show that a person who identifies closer to the political right does not justify corruption more or less relative to those identifying closer to the political left. However, the interaction term shows that people answering higher values of this variable justified corruption more in 2016. Once again, the significance of the year dummy is lost when considering this variable. With a higher number of respondents identifying with the political right wing, who appear to justify corruption more, it would be understood how overall corruption tolerance increased.

% I need to draw the graph which shows the visual differences between groups and their corruption tolerance

% Now I do the data wrangling needed for this


% Now do the graph
\begin{figure}[htbp!]
\begin{knitrout}
\definecolor{shadecolor}{rgb}{0.969, 0.969, 0.969}\color{fgcolor}

{\centering \includegraphics[width=\maxwidth]{figure/difgraph-1} 

}


\end{knitrout}
\caption{Graphical representations of corruption tolerance across key explanatory variables}
\label{fig:difgraph}
Figures show the percent that justify corruption across the groups used as explanatory models in Table \ref{tab:simplemodel}. Error bars represent the 95\% confidence intervals adjusted for design effects.
\end{figure}

These findings are supported by Figure \ref{fig:difgraph}. According to panel (a), in 2014, only 12.03\% of those not unemployed justified corruption, while in 2016 this figure increased to 27.03\%, very close to the percentage of unemployed people who justified it in 2016. The time difference between these point estimates is not statistically significant, which means that in 2016 the effect of unemployment in corruption tolerance approached zero. Thus, Figure \ref{fig:difgraph} along with Model 2 of Table \ref{tab:simplemodel} show that it was not the unemployed who started to justify corruption less, it was that the people who were not unemployed started to justify it more. Panels (b) and (c) of Figure \ref{fig:difgraph} show that the percentage of people who either confided in or approved the President and justified corruption increased significantly  between 2014 and 2016. This means that the negative effect of supporting the executive in 2016 was smaller than in 2014, as confirmed by the interaction term in Models 3 and 4 of Table \ref{tab:simplemodel}. In panel (d) of Figure \ref{fig:difgraph}, where four different political groups are considered: the left, right, center and those who did not answer the question. All four groups saw increases in the percent of group members who justify corruption. All increases in corruption tolerance are significant, except for those who identify with the left wing. This is consistent with the coefficient sign seen in Model 5 for the political score variable. 


% Results 2 Child Document


% Results II .Rnw File


Now the general model as described by Equation \ref{eqn:genmod} is estimated with the key regressors as well as a set of controls at the individual level. I keep the variables which yielded statistically significant interaction tears with the year dummy, except for confidence in the president as job approval ratings contemplated the same effects. Coefficients are shown in Table \ref{tab:complexmod} and average partial effects are shown in Table \ref{tab:apescomp}. 

% Estimate the modified models, by sourcing that .R file


% Now create the table
\begin{table}[htbp]
\begin{center}
\caption{Logit coefficients for modified models}
\label{tab:complexmod}

\begin{tabular}[t]{lccc}
\toprule
  & Model 1 & Model 2 & Model 3\\
\midrule
Constant & \num{-0.674}* & \num{0.707} & \num{-0.351}\\
 & (\num{0.401}) & (\num{0.468}) & (\num{0.405})\\
2016 Dummy & \num{0.887}*** & \num{-1.217}** & \num{0.333}\\
 & (\num{0.145}) & (\num{0.477}) & (\num{0.252})\\
Woman & \num{0.124} & \num{0.136} & \num{0.127}\\
 & (\num{0.109}) & (\num{0.111}) & (\num{0.109})\\
Age & \num{-0.026}*** & \num{-0.026}*** & \num{-0.026}***\\
 & (\num{0.004}) & (\num{0.004}) & (\num{0.004})\\
Years of education & \num{-0.041}*** & \num{-0.038}** & \num{-0.039}**\\
 & (\num{0.015}) & (\num{0.015}) & (\num{0.015})\\
Lives in urban setting & \num{-0.020} & \num{0.013} & \num{0.009}\\
 & (\num{0.132}) & (\num{0.131}) & (\num{0.132})\\
External political efficacy & \num{-0.047} & \num{-0.041} & \num{-0.044}\\
 & (\num{0.032}) & (\num{0.032}) & (\num{0.032})\\
Internal political efficacy & \num{0.096}** & \num{0.093}** & \num{0.089}**\\
 & (\num{0.041}) & (\num{0.042}) & (\num{0.041})\\
Participation in a protest & \num{0.431}** & \num{0.450}** & \num{0.471}**\\
 & (\num{0.204}) & (\num{0.205}) & (\num{0.207})\\
Interest in politics & \num{-0.249}** & \num{-0.220}* & \num{-0.244}**\\
 & (\num{0.116}) & (\num{0.119}) & (\num{0.119})\\
Perceptions of corruption & \num{0.000} & \num{0.001} & \num{-0.033}\\
 & (\num{0.133}) & (\num{0.137}) & (\num{0.136})\\
Exposure to corruption & \num{0.985}*** & \num{1.003}*** & \num{1.008}***\\
 & (\num{0.115}) & (\num{0.114}) & (\num{0.115})\\
Unemployment & \num{0.956}*** & \num{0.296}** & \num{0.285}*\\
 & (\num{0.215}) & (\num{0.146}) & (\num{0.145})\\
Approval of Pres. Performance & \num{-0.132}** & \num{-0.510}*** & \num{-0.128}**\\
 & (\num{0.063}) & (\num{0.102}) & (\num{0.063})\\
Political Wing & \num{0.028} & \num{0.029} & \num{-0.025}\\
 & (\num{0.020}) & (\num{0.019}) & (\num{0.040})\\
Unemployment Interaction & \num{-0.908}*** &  & \\
 & (\num{0.275}) &  & \\
Pres. Approval Interaction &  & \num{0.543}*** & \\
 &  & (\num{0.122}) & \\
Pol. Wing Interaction &  &  & \num{0.081}*\\
 &  &  & (\num{0.046})\\
\midrule
$N$ & \num{2308} & \num{2308} & \num{2308}\\
AIC & \num{2201.72} & \num{2191.11} & \num{2208.60}\\
BIC & \num{2301.92} & \num{2290.64} & \num{2307.42}\\
\bottomrule
\end{tabular}


\end{center}
Logit coefficients of the modified models as described by Equation \ref{eqn:genmod} with design-adjusted std. errors. *$p$ < 0.1, **$p$< 0.05, ***$p$ < 0.01.
\end{table}

% APE table
\begin{table}[htbp]
\begin{center}
\caption{Average partial effects for models in Table \ref{tab:complexmod}}
\label{tab:apescomp}

\begin{tabular}[t]{lccc}
\toprule
  & Model 1 & Model 2 & Model 3\\
\midrule
Age & \num{-0.004}*** & \num{-0.004}*** & \num{-0.004}***\\
 & (\num{0.001}) & (\num{0.001}) & (\num{0.001})\\
Years of education & \num{-0.006}*** & \num{-0.006}** & \num{-0.006}***\\
 & (\num{0.002}) & (\num{0.002}) & (\num{0.002})\\
External political efficacy & \num{-0.007} & \num{-0.006} & \num{-0.007}\\
 & (\num{0.005}) & (\num{0.005}) & (\num{0.005})\\
Internal political efficacy & \num{0.015}** & \num{0.014}** & \num{0.014}**\\
 & (\num{0.006}) & (\num{0.006}) & (\num{0.006})\\
Interest in politics & \num{-0.038}** & \num{-0.033}* & \num{-0.037}**\\
 & (\num{0.018}) & (\num{0.018}) & (\num{0.018})\\
Perceptions of corruption & \num{0.000} & \num{0.000} & \num{-0.005}\\
 & (\num{0.020}) & (\num{0.021}) & (\num{0.021})\\
Exposure to corruption & \num{0.150}*** & \num{0.152}*** & \num{0.154}***\\
 & (\num{0.017}) & (\num{0.017}) & (\num{0.018})\\
Unemployment & \num{0.055}*** & \num{0.045}** & \num{0.044}*\\
 & (\num{0.020}) & (\num{0.022}) & (\num{0.022})\\
Approval of Pres. performance & \num{-0.020}** & \num{-0.023}** & \num{-0.019}**\\
 & (\num{0.010}) & (\num{0.009}) & (\num{0.010})\\
Political wing & \num{0.004} & \num{0.004} & \num{0.004}\\
 & (\num{0.003}) & (\num{0.003}) & (\num{0.003})\\
\midrule
$N$ & \num{2308} & \num{2308} & \num{2308}\\
\bottomrule
\end{tabular}


\end{center}
Average partial effects for models in Table \ref{tab:complexmod}, with design-adjusted std. errors. *$p$ < 0.1, **$p$< 0.05, ***$p$ < 0.01.
\end{table}

These models include multiple control variables suggested by \textcite{Moscoso.2020} and \textcite{Lupu.2017}. Of these, only age is significant and has a negative effect on corruption tolerance. A person older by one year is 4 percentage points less likely to justify corruption. Political efficacy indicators are included too. The external political efficacy question, which asks if respondents believe that politicians serve the interests of the people, has no statistical significance. Internal political efficacy asks about how well the respondent understands politics and this control is significant; a person who understands more about the country's politics is more likely to justify corruption, the estimated increase in corruption tolerance probability is about 1.5 percentage points.

While \textcite{Moscoso.2020} find that none of the political efficacy variables are significant for corruption tolerance in 2019, they find that interest in politics is significant and has a positive effect. That finding is reversed here: interest in politics is significant yet portrays a negative relationship between the two: more interest in the country's politics is actually negatively related with corruption tolerance. A person who reports being interested in politics is about 3.5 percentage points less likely to justify corruption. While they may appear to ask similar things, the two questions may imply different attitudes to politics: the political efficacy question asks if citizens are politically aware, and the second one asks if they are interested in entering politics. Separating these two questions may imply that attitudes of apathy or pragmatism to the political society are separated from an \enquote{idealist} attitude towards it of those who would like to enter politics.

A control for years of education is also added and it is significant, communicating that more educated respondents are less likely to justify corruption. Other things equal, an additional year of education is related to a 6 percentage points reduction in corruption tolerance. This finding is intuitive considering that more education may mean more knowledge about the costs of corruption. The social payoffs for being honest may be higher as also higher education may entail a better economic position which makes engaging in corrupt acts less economically attractive. 

Exposure to corrupt acts (paying or being asked to pay a bribe) is also strongly correlated with tolerance. A person who has been exposed to some form of bribing is about 15\% more likely to justify corruption. The causality directon is not clear as it might be possible that a predisposed tolerance to corruption due to external factors makes citizens more likely to be in environments where corruption flourishes. This may suggest that younger people justify corruption partly because they are more exposed by it: empirical models not shown explicitly here show that an interaction term between age and corruption exposure is significant at the the 90\% confidence level. Corruption perceptions, on the other hand, play no role in determining corruption tolerance for this time period. 

% Now the regressions which were not shown explicitly in text


A dummy variable equal to unity for respondents who have recently attended a protest is added and it is very significant. Other things equal, a person who has attended a protest is about 7\% more likely to justify corruption. This might be related to \textit{denial of victim} explanation as proposed by \textcite{Ashforth.2003}. People who attend protests probably reject the current state of things, which may induce a feeling of contempt against society. They may believe dishonest acts could be justified in these circumstances because they feel corrupt acts can be \enquote{retribution} by alleging that \textit{small} corruption acts are nothing compared to grand corruption scandals. Since they have \enquote{declared} their rejection to the system in general, they have surrendered to its flaws and have no social incentives to remain honest.

Most importantly, Table \ref{tab:complexmod} shows that after considering several variables suggested by the literature, the interaction terms as estimated in Table \ref{tab:simplemodel} keep their sign and significance. It is still true that unemployed respondents justified corruption more in 2014 and less in 2016. People who approved the job performance of the President were less likely to justify corruption in both years, but their rejection was smaller in 2016. Finally, while political identification was not significant in 2014, it was in 2016, where people who identified as closer to the political right were more likely to justify corruption. 

It is possible that those initially unemployed justified corruption more because it was their \enquote{steady state} of corruption tolerance: unemployed people are economically disadvantaged which gives them incentives to engage in corrupt actions which can yield positive economic payoffs. Additionally, as they are unable to enter the job market, they might feel alienated from society, which might decrease social or moral incentives to remain honest by renouncing the economic payoffs that corruption may offer. The change in corruption tolerance for 2016 is more difficult to understand. It is possible that, since the recession, many have lost jobs and have had relatively short unemployment spells. The recently unemployed may not feel too alienated from society and thus have not adopted an attitude of pragmatism toward the current circumstances. Savings or family income may support the recently unemployed which makes them feel less desperate and prone to take the \enquote{moral high ground}. This all contributes to them still feeling part of society, which reduces their rationalization of corruption. However, with larger unemployment spells, desperation may trigger more pragmatic points of view, which will lead to higher corruption tolerance in the future.

To better understand the implications of the political variables' coefficients and their change in time, consider a key effect on corruption normalization, leadership. Having initially branded himself as \enquote{the biblical underdog} \parencite[para. 4]{Hedgecoe.2009}, President Correa distanced himself from the country's political elite and constantly denounced corruption and injustice in the system. The new government promised a radical change in 2007 and did deliver in a way as it gave Ecuador a politically stable though totalitarian environment, as well as other changes in political and economical mechanisms \parencite{Weisbrot.2017}. The President had explicitly stated that he would battle corruption and fiscal evasion \enquote{to the death} \parencite{Ortiz.2013}. Therefore, supporters of the regime faced higher social sanctions when justifying corrupt behavior, as this may have implied that the economic and political model they supported was flawed.

However, by 2016 the popularity of the government saw a sharp decrease. Several narratives started to be constructed by President Correa and his officials to explain the flaws and weaknesses that opponents had denounced. These included reducing corruption accusations to \enquote{political persecution} or unfounded claims done because of upcoming elections \parencite{Melendez.2017}. A statement by the President represents a particularly relevant example: a regime-affiliated newspaper portrayed how Correa qualifies the Panama Papers as a \textit{selective fight against corruption} which is nothing but another kind of corruption, as well as a \enquote{strategy by power groups to destabilize democratically-elect governments} \parencite[para. 5-7]{Telegrafo.2016}. 

Even as corruption accusations had planted the seed of a deep investigation about a complex corruption scheme involving top government officials and large corporations \parencite{Villavicencio.2019}, authorities within the Correa administration reduced the importance of these events, which created a narrative for regime supporters. If the legitimacy of those who denounce and control corruption is questioned by an important authority of the organization, corrupt acts can be more easily normalized \parencite{Ashforth.2003}. Thus, if there was a greater incidence of corrupt acts as well as numerous attempts by the authorities to justify them, it can be understood how supporters of the regime started to justify corruption more.

Results also show how people who identified with the political right became more corruption-tolerant in 2016. It is not clear if there is a causal relationship involving the political right and corruption tolerance. This is because it has been determined that in Ecuador the answer to the political identification question has little to do with the traditional principles of the political wings. Rather, it is possible that the political self-identification of Ecuadorians follows a multidimensional perspective \parencite{Moncagatta.2020b}, not too accurately measured with an indicator like the one used here. 

A potential explanation to the sign on this variable is that those who identify with the right do so partially because they consider themselves to be against the ruling government. This is reasonable considering the increase in the percentage of \enquote{rightist} from 2014 to 2016, which moves together with regime's downfall. Additionally, it is possible that anti-regime attitudes formed under a common set of ideas rather than under a political party or figure, since during President Correa's tenure opposition forces did not materialize strongly behind a party or leader \parencite{Melendez.2017}. It is sensible to believe that no political wing has any particular preference for justifying or rejecting corruption, as important academics \parencite{Holcombe.2015} and politicians \parencite{Morris.2021} associated with both political wings have denounced corruption. Anti-regime respondents rather than those actually identified with the political right might rationalize corruption as a form of retribution, as proposed by \textcite{Ashforth.2003} and discussed by \textcite{Adoum.2000} in the Ecuadorian case.

Some limitations are worth discussing. One of the most important issues is the possible differences across individuals in their understanding of \enquote{bribes}. Even though the EXC18 question mentions \textit{paying a bribe} it is possible that the idea that comes to mind to respondents is outside the mentioned hypothetical situations; what respondents think when hearing \textit{paying a bribe} could vary. This implies that observations are not homogeneous. Another issue is the social desirability bias: the corruption tolerance variable may be considerably mismeasured due to this, and social desirability bias incidence may be heterogeneous across unobserved characteristics correlated to our key regressors.



% Conclusions Child Document


% Conclusions .Rnw File


\section{Conclusions}

The degree to which citizens of a country justify corruption is a topic worth of careful study given that the more corruption is normalized, the more likely it is that actors in that environment commit it. This is because corruption necessarily implies both social and economic payoffs, and when the social payoff of being honest is eliminated through a justification of dishonest acts, the economic payoffs now almost fully drive the decision of an individual to engage in these. In Ecuador, the data of the AmericasBarometer (AB) survey has shown that corruption tolerance has risen since 2014, the most important increase being between 2014 and 2016. Binary-outcome logit models are implemented to find the determinants of this increase, which relates the probability of tolerating corruption to several individual-level indicators from the AB. It is determined that changes in presidential job approval as well as in political wing preferences during the 2014 and 2016 period could have influenced the corruption tolerance increase. It is also found that those not unemployed justified corruption more in 2016 relative to those who were unemployed.

The percent who reports being unemployed, which has risen from 2014 to 2016, stood out as an initial key regressor as well. While an interaction term of unemployment with a 2016 dummy is significant, it is not evidence that this variable drove the rise. The interaction coefficient is negative, meaning that people who were unemployed in 2016 justified corruption less than in 2014. Although it does not explain the jump in corruption tolerance, it is still an interesting finding since it implies that the new unemployed respondents behave differently than unemployed respondents in a non-recession year like 2014: the newly unemployed do not feel as alienated from the system as the people with longer unemployment spells.

Variables which measure the degree to which respondents support the President, which show sharp decreases in the period, are identified as key drivers of the increase. The data show that people who approved the President's performance were less likely to justify corruption in 2014. However, this changed in 2016 as the interaction term with the year dummy was positive and significant, meaning this group of people justified corruption more in 2016. This shift in attitude may be evidence of mechanisms of rationalization of corruption. In the heyday of President Correa's administration, the regime kept a ruthless narrative of against corruption. However, as economic and political conditions started to deteriorate after 2014, regime leaders created rationalization narratives, in which whistleblowers of corrupt acts were often seen as illegitimate and legal instances involving acts of corruption were often dismissed as political persecution. The role of authorities in engaging in corrupt acts and later rationalizing them may have had a role in institutionalizing and socializing corruption among the supporters of the regime, as proposed by \textcite{Ashforth.2003}. 

The percent who report being identified with the \enquote{right} political wing of the left-right dichotomy, which has risen significantly since 2014 is another key driver. It was found that in 2016 a person who identified closer to the right wing was more likely to justify corruption. This was not seen in 2014, which is why only the interaction term with year is significant. It is possible that people who started to identify with the right in 2016 do so because they are against the current administration. If this were to be the case, it is possible that the reason why the political wing variable is a significant driver of the corruption tolerance is that this group feels distanced with the government and rationalizes acts of corruption as retributions against them. If these people perceive that the current government is corrupt or unable to manage the nation, they might engage in rationalizations which justify corrupt acts. However, since there are many issues with the political identification methodology of people as the question may be understood differently across subjects, this conclusion should be researched further.

Considering this empirical evidence, the jump in corruption tolerance between 2014 and 2016 can be explained. The economic recession brought about by the collapse of commodity prices, the dependence of government expenditure and the earthquake of April 2016 combined with the numerous accusations of corruption against government officials deteriorated regime support. This led to a decrease in the number of people who approve the President and an increase in the number of people who identify with the political right. This represented a decrease of the people who did not justify corruption and an increase of people who did, thus accounting for the significant increase of corruption tolerance.

The most robust findings of the literature are confirmed. Exposure to corruption is a strong predictor for corruption tolerance, where people exposed to bribes are more likely to justify corruption. The direction of causality between is not clear. Also, age is a negative predictor of corruption tolerance, a troubling finding which potentially exposes a flawed education system and little attention to the political inclusion of younger citizens. This paper also includes years of education as a control, and it is found that it is only significant for 2016 as a negative predictor. Education and the way it is carried out may have a significant effect on how people behave toward dishonest behavior as pointed out by \textcite{Adoum.2000} who considers academic dishonesty as a precedent for political corruption. 

This paper's findings suggest obscure details about the way that Ecuadorians behave around corruption. The considerable amount of consequences of corruption in the last years have not made the people tired of dishonesty. In fact, it seems that it has only made them more willing to engage in it. The opposition groups of President Correa's regime, which often cite corruption scandals as arguments against left-leaning politicians, have seemingly become more open to the idea that corruption is inherent to politics and that it can be justified it if suits their needs. Something similar can be argued about the people who participate in protests who are found to be other sources of corruption tolerance. Nevertheless, this phenomenon is not isolated to opposition groups, it is also found among regime supporters. When corruption became the norm among leaders, supporters became pragmatic with corrupt acts. What both of these possible lines of reasoning entail is that corruption will keep happening regardless of who is in power, as both parts in politics have found the way to allow deceit to exist. Calls for honesty have been bent to a point that it they become devoid of true meaning, only used if such honest works to the convenience of those speaking about it. 

The costs of corrupt behavior are well documented in the literature: they challenge the validity of democratic systems \parencite{Moscoso.2018}, destroy wealth, distort markets as well as hinder economic growth and income distribution (\textcite{Shleifer.1993}, \textcite{Singer.2016}). Corruption can even add to human misery through shorter life expectancy \parencite{Siverson.2014}, a result that can be expected to crudely appear in Ecuador soon, considering an extensive corruption incidence during the COVID-19 pandemic. The problem of corruption, as clearly pervasive as it may appear, is a politically and emotionally charged discussion topic, up to the point that the truth can appear blurry. While policymaking and legal action might be ways to change attitudes toward corruption, it will be difficult to fully eliminate corruption solely through this way. It is the philosophy of honesty by convenience that must be vanquished through individual action and reflection, so that dishonesty can be reprehended enough to conspicuously influence social incentives and escape the atrocious evils that corruption espouses. 

\end{document}
